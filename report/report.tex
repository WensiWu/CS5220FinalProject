\documentclass[11pt]{article}
\usepackage{amsmath,amsthm,verbatim,amssymb,amsfonts,amscd, graphicx}
\usepackage{graphicx}
\usepackage{subcaption}
\usepackage{listings}
\usepackage{float}
\usepackage{url}
\usepackage{titlesec}
\setcounter{secnumdepth}{4}


\graphicspath{{./times/}}
\topmargin0.0cm
\headheight0.0cm
\headsep0.0cm
\oddsidemargin0.0cm
\textheight23.0cm
\textwidth16.5cm
\footskip1.0cm

\titleformat{\paragraph}
{\normalfont\normalsize\bfseries}{\theparagraph}{1em}{}
\titlespacing*{\paragraph}
{0pt}{3.25ex plus 1ex minus .2ex}{1.5ex plus .2ex}


\begin{document}
\title{CS 5220\\ Final project - Computing Truss (?)}
\author{Marc Aurele Gilles (mtg79)\\ Wensi Wu(382) }
\maketitle

\section{Introduction}


\section{Problem Description}

We are computing truss TADATADA - can probably copy paste things from earls notes


\section{Numerical Method}

The truss computing problem can be written in the form

$$F_l(u)=0$$
where u is the truss, and F is a system of $n$ non-linear equations, where n depends on the number of nodes, elements and degrees of freedom(?), and l is the maximum load factor.

We solve this system of equation using a continuation method, starting at $l=0$, and incrementally increasing $l$ until it reaches the user specified maximum $l^{\star}$. Each $F_l(u)=0$ equation is solved by a Newton method iteration.
\\
That is we repeatedly solve $F_l(u_{t+1}) \simeq F(u_t) + S(u_t)*u_{t+1} =0$, where $T(u_t)$ is the Jacobian matrix at $u_t$, which in this problem is the same as the Stiffness matrix.\\

In other words, each iteration is a linear solve:
$S(u_t)*u_{t+1}=-F(u_t)$.

Most of the computation of this program is spent doing this linear solve. The stiffness matrix is very high dimensional for large structures, but also very sparse, a sparse solver is therefore needed.

\section{Setup: generating input structures}

As our objective is to speed up this computation, and conduct scaling studies, we need to be able to generate input structures of variable sizes.
We generated two different type of input structures, a "chain" structure, and a "pyramid" structure.
\\
Here put drawings
\\
For each structure we declare the position, of each node, the position of each elements (as defined by a pair of nodes), the properties of each element (cross section area and ... ), and a load factor on each node.

These two different structure give rise to different sparsity patterns in the stiffness matrix. The chain structure produces a strictly diagonally dominant structures, which isn't strictly the case for the pyramid structure.
\\
add plots of sparsity structure

\section{Original code}

\subsection{Skyline}

\subsection{Timing of original code}

\section{New code}

\subsection{Pardiso}

\subsection{Compressed sparse row}

\section{Scaling studies}

\section{Conclusion}

\end{document}